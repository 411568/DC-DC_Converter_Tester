\chapter{Cel pracy}
\label{chapter0}

Celem pracy jest zaprojektowanie, wykonanie i przetestowanie urządzenia do pomiarów parametrów przetwornic DC-DC. 
Urządzenie powinno spełniać następujące warunki:

\begin{itemize}
    \item możliwość pomiaru sprawności przetwornic, regulacji napięcia wyjściowego, zakresu napięć wejściowych i prądów obciążenia,
    \item modularność konstrukcji,
    \item możliwość dodania kolejnych modułów, umożliwiających wykonywanie innych pomiarów,
    \item łatwość modyfikacji sprzętu i oprogramowania,
    \item niska cena w porównaniu do rozwiązań komercyjnych.
\end{itemize}

W celu spełnienia przedstawionych powyżej założeń, zaprojektowane urządzenie musi posiadać:
\begin{itemize}
    \item zintegrowany zasilacz regulowany,
    \item zintegrowane obciążenie aktywne,
    \item wbudowany kontroler, pozwalający na pracę bez konieczności podłączania do komputera, wyposażony w podstawowy interfejs użytkownika z wyświetlaczem.
\end{itemize}


W rozdziale \ref{chapter-1} przedstawiono budowę i rodzaje przetwornic DC-DC.

W rozdziale \ref{chapter-2} zaprezentowano rozwiązania komercyjne urządzeń do pomiaru parametrów przetwornic. Opisano 
również zasadę działania projektowanego urządzenia.

W rozdziale \ref{chapter-4} zawarto opis realizacji projektu, zarówno pod kątem elektroniki, oprogramowania, jak i obudowy.

W rozdziale \ref{chapter-5} przedstawiono testy skonstruowanego urządzenia.

