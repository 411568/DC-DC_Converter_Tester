\FloatBarrier
\chapter{Podsumowanie}
\label{chapter-6}

Celem niniejszej pracy było zaprojektowanie i wykonanie modułowego urządzenia, umożliwiającego automatyzację pomiarów 
przetwornic napięciowych (DC/DC). W tym celu konieczne było zaprojektowanie zasilacza regulowanego, obciążenia aktywnego
oraz kontrolera, przy pomocy którego możliwe byłoby zarządzanie innymi modułami.
Moduły te zostały wykonane, a następnie przetestowane. Stwierdzono poprawne działanie wszystkich elementów oraz zadowalającą
dokładność stworzonego sprzętu. 

Automatyczny pomiar sprawności przetwornicy wykorzystywany był wielokrotnie z pozytywnym wynikiem i znacząco przyspieszał
proces. Napisane oprogramowanie ocenione zostało przez osoby testujące jako responsywne i proste w użyciu. 
Wykonanie stosownej obudowy z odpornych materiałów, oraz przemyślany interfejs zapewniają sprawne działanie bez obaw 
o ewentualne uszkodzenia czy awarie.
Zrealizowany projekt inżynierski wykorzystuje wiele elementów, które wybrane zostały przez autora pracy z myślą o dalszym rozwoju
i potencjalnym modyfikacjom. W ramach tego, możliwe jest dodanie dodatkowych funkcjonalności automatycznego pomiaru innych 
parametrów przetwornic, czy też innych urządzeń, np.: pomiar pojemności akumulatorów, pomiar rezystancji przewodów metodą
Kelvina.
 